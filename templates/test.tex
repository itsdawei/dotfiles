\documentclass[a4paper]{/home/dawei/.dotfiles/templates/preamble}

\cfoot{\thepage}
\lhead{Lee}
\chead{CSCI-670: Advanced Analysis of Algorithms}
\rhead{Homework 3}

\linespread{1.5}

\begin{document}
\textit{I collaborated with Aditya Prasad. I am only auditing this class, so
please grade my submission after all the other ones. Thanks!}

\begin{theorem}{Best Theorem}{}
  This the best theorem.
\end{theorem}

\begin{corollary}{A Corollary of the Best Theorem}{}
  This a corollary of the best theorem.
\end{corollary}

\begin{remark}{}{}
  This is a remark.
\end{remark}

\begin{lemma}{A Minor Lemma}{}
  This is a lemma.
\end{lemma}

\begin{example}{An Example}{}
  This is an example.
\end{example}

\noindent\textbf{Problem 1} \\
For notation, let $b(T) = \sum_{e \in T} b_e$. Recall that $b(T^*) = b(T'_e)
\iff b(T^* \setminus T'_e) = b(T'_e \setminus T^*)$. Furthermore, let $f$ be the bijection between $T^* \setminus T'_e$ and $T'_e \setminus T^*$,
we have
\[
  \sum_{e \in T^* \setminus T'_e} b_e = \sum_{e \in T'_e \setminus T^*} b_e = \sum_{e \in T^* \setminus T'_e} b_{f(e)}
\]
Let us select some $i \in T^* \setminus T'_e$. By the strong exchange property,
$T_i = T^* \setminus \{i\} \cup \{f(i)\}$ is a feasible solution. Since we know
that $T^*$ is a min-cost bases in $b$-values, $b(T^*) \le b(T_i) \iff b_i \le
b_{f(i)}$. Note that our construction did not depend on the $i$ that we picked,
i.e.\ $b_i \le b_{f(i)}$ for all $i \in T^* \setminus T'_e$.

Finally, combining this with the equation above $\sum_{e \in T^* \setminus
T'_e} b_e = \sum_{e \in T^* \setminus T'_e} b_{f(e)}$, we know that $b_i
= b_{f(i)}$ for all $i \in T^* \setminus T'_e$. In fact, this extends to all $i
\in T^*$ as we can define $f(i) = i$ for all $i \in T^* \cap T'_e$.

Now, let us examine $p_e = c_e + c(T^*_e) - c(T^*)$ for some $e \in T^*$. Let
$T_e = T^* \setminus \{e\} \cup \{f(e)\}$ for $e \in T^* \setminus T^*_e$. Note
that, since $T^*_e$ is a min-cost feasible set that does not include $e$, it
must be that $c(T^*_e) \le c(T_e)$. Further, we have the following inequality
\begin{align*}
  p_e &\le c_e + c(T_e) - c(T^*) \\
      &\le c_e + \sum_{i \in T^* \setminus \{e\} \cup \{f(e)\}} c_i - \sum_{i \in T^*} c_i\\
      &\le c_e + c_{f(e)} - c_e \\
      &\le c_{f(e)}
\end{align*}
so the value $p_e$ is upper-bounded by $c_{f(e)}$.
We can likewise extend this arugment to all $e \in T^*$ by defining $f(e) = e$
for $e \in T^*$. By definition, $b_{f(e)} \ge c_{f(e)}$. Thus, we have the
following inequality for all $e \in T^*$
\[
  b_e = b_{f(e)} \ge c_{f(e)}
\]
Therefore,
\[
  \sum_{e \in T^*} b_e \ge \sum_{e \in T^*} c_{f(e)} \ge \sum_{e \in T^*} p_e
\]

\noindent\textbf{Problem 2}
\begin{enumerate}[label = (\alph*)]
  \item Assume that there exists $e' = (u',v') \in E$ s.t. $w'_{e'} = w_{e'} + b_{u'} - b_{v'} < 0$.
    This implies that
    \[
      w_{e'} + b_{u'} < b_{v'}
    \]
    Contradiciton because $b_{v'}$ is the shortest-path distance from $s$ to $v'$ and
    the above inequality implies that the shortest-path to from $s$ to $u'$ and the edge $e'$ is
    a shorter path to $v'$.
  \item For some pair of nodes $(v_0,v_k)$ in the original graph, the shortest-path
    distance $d_{v_0,v_k}$ with respect to the new weights $w'_e$ is given by
    \[
      d_{v_0,v_k} = w'_{v_0, v_1} + w'_{v_1, v_2} + \cdots + w'_{v_{k-1},v_k}
    \]
    where the shortest path from $v_0$ to $v_k$ is given by $(v_0, v_1, \dots, v_k)$.
    Let us expand the above sum
    \[
      d_{v_0,v_k} = w_{v_0, v_1} + b_{v_0} - b_{v_1} + w_{v_1, v_2} + b_{v_1} - b_{v_2} + \cdots + w_{v_{k-1}, v_k} + b_{v_{k-1}} - b_{v_k} 
    \]
    Note that we have a telescoping sum, which can be simplified as follows.
    \[
      d_{v_0,v_k} = w_{v_0, v_1} + w_{v_1, v_2}  + \cdots + w_{v_{k-1}, v_k} + b_{v_0} - b_{v_k} 
    \]
    Subtracting $(b_{v_0} - b_{v_k})$ from the above equation we get the shorest-path
    distance with respect to the original edge weights.
\end{enumerate}

\noindent\textbf{Problem 3}
  \begin{enumerate}[label = (\alph*)]
  \item Assume that the set of chosen edge $e_v$ forms a cycle $C$ of length
    $k$, i.e.\ there are $k$ nodes and $k$ edges on the cycle. Let $e^*
    = (a,b)$ be the cheapest edge in $C$. Note that $e_a = e_b = e^*$ because
    $e^*$ is the cheapest edge on the cycle. Since there are one more node
    than edges, by the pigeonhole principle, two edge must be picked by the
    same node. Contradiction --- one node can only pick one edge.
  \item 
    \begin{proof}[The algorithm produces an acyclic graph]
      Assume, for contradiction, that the algorithm produces an acyclic graph
      at iteration $i$. Let $E_{i}$ be the set of all edges selected by the
      algorithm up until iteration $i$, and let $S$ be the set of edges
      selected on iteration $i$. Without loss of generality, we can assume that
      $E_{i-1}$ is acyclic. Notice that $E_{i-1}$ is also defining a set of
      connected components (contracted nodes). $E_i = E_{i-1} \cup S$. Since
      $E_{i-1}$ is acyclic, all the connected components are acyclic. Thus, the
      only ways that a cycle could be created is for $S$ to create a path from
      connected component $A$ to connected component $B$ and back to $A$.
      However, this will be a cycle in $S$, which we have shown to be not
      possible.
    \end{proof}
    \begin{proof}[The algorithm produces an connected graph]
      The algorithm only terminate when there is only one connected component left.
      This implies that the output graph is connected.
    \end{proof}
    The above two proof implies that the algorithm produces a spanning tree.
    \begin{proof}[The algorithm produces an MST]
      The edge picked by each node satisfies the cut property. Specifically,
      for a connected component $C$ defined by the contracted node $v$, the
      edge $e$ picked by $C$ is the minimum edge across the cut $(C, \bar{C})$.
      Hence, the spanning tree composed of all such edges will be a MST by the
      cut property.
    \end{proof}
  \item We will make use of the Union-Find data structure. We will maintain
    a list of connected components.
    \begin{algorithmic}
      \State{CC $\gets$ each node in the graph as a connected component}
      \While{$|CC| > 1$}
        \State{edgeList $\gets$ []}
        \For{$v$ in CC}
          \State{$e \gets$ the cheapest incident edge of $v$.}
          \State{add $e$ to edgeList}
        \EndFor{}
        \For{$e = (u,v)$ in edgeList}
          \State{x = Union (u,v)} \Comment{We assume that Union returns the new root of the CC.}
          \State{without loss of generality, let $u$ be the new root.}
          \State{reassign the edges s.t.\ all edges $(v,a)$ is now $(u,a)$.}
        \EndFor{}
      \EndWhile{}
    \end{algorithmic}
    Note that, in this algorithm, all edges leaving a connected component will be
    connected to the root of the connected component.
    \begin{proof}[Runtime Analysis]{}
      It takes $O(m)$ to find the cheapest edge for all of the connect
      component ($n \cdot \deg{v}$). It takes $O(m)$ to reassign all the edges
      to the new root ($n \cdot \deg{v}$). Each iteration of the outer-most
      while loop reduce the number of connect component by a factor of 2, so
      the while loop runs $O(\log{n})$ times. Hence, the algorithm takes $O(m
      \log{n})$. Further, since $n \le m$, we have $O(m \log{m})$.
    \end{proof}
\end{enumerate}

\noindent\textbf{Problem 4}
\begin{description}
  \item[insert $O(1)$] The runtime of Insert does not change; we still just add
    a new tree.
  \item[delete-min $O(\max{\mathrm{rank}})$] The runtime of Delete-Min does not
    change. We still need to check all the roots after performing
    \texttt{clean-up}. The only question is whether this new removal scheme
    ensures at most $O(\log n)$ roots for our heap.
  \item[decrease-key $O(1)$] The runtime of Decrease-Key does not change. We
    cut out the subtree rooted at the node that we decrease. Furthermore, as
    the new removal rule states, if this node is the third lost child of
    a parent, then we remove the parent as well (apply the same rule
    recursively to the grand-parants).
\end{description}
What we are concerned with is whether the amortized runtime of
\texttt{decrease-key} remain $O(\log n)$.

Following a similar analysis in the lecture, we define $S_k$ to be the smallest
number of node in a tree of rank $k$. In our new scheme, $S_0 = 1$ because
$S_0$ only consists of the root itself. $S_1 = 2$ because a tree with one child
has size 2. Note that in the new removal scheme,
\[
  S_k = 1 + S_0 + S_0 + S_0 + S_1 + S_2 + \cdots + S_{k-4} + S_{k-3}
\]
because the smallest number of nodes in the tree of rank $k$ is given in the
scenario where every subtree loses two children (if they have at least two
children, otherwise lose all their children). By a similar argument, we have
\[
  S_{k-1} = 1 + S_0 + S_0 + S_0 + S_1 + S_2 + \cdots + S_{k-5} + S_{k-4}
\]
Then, $S_k - S_{k-1} = S_{k-3}$ and $S_{k} = S_{k-1} + S_{k-3}$. Let us assume
that $S_k$ is exponential in $k$, i.e. $S_k = c^k$ for some $c > 1$.
We have that $c^k = c^{k-1} + c^{k-3}$ and $c^3 = c^2 + 1$. By plugging this
equation into a cubic solver, we find that the only real root is $c = 1.466$.
Note that it suffice to show that $S_k \ge c^k$, so we will do so via
a induction proof.
\begin{lemma}{}{}
  $S_k \ge c^k$ where $c = 1.466$ is the solution to $c^3 = c^2 + 1$.
\end{lemma}
\begin{proof}{}{} \\
  B.C.: For $k = 0$, $S_0 = 1 \ge 1$. For $k = 1$, $S_1 = 2 \ge 1.466$. \\
  I.H.: The lemma holds for $0 \le n \le k$. \\
  I.S.: Consider $k+1$, we know that $S_{k+1} = S_{k} + S_{k-2}$.
  By our I.H., $S_k \ge c^k$ and $S_{k-2} \ge c^{k-2}$.
  Hence, 
  \[
    S_{k+1} \ge c^k + c^{k-2} = c^{k-2}(c^{2} + 1) = c^{k-2} \cdot c^{3} = c^{k+1}
  \]
\end{proof}
Since the max rank of our fibonacci heap remains logarithmic in terms of the number of nodes,
it follows that the runtime of \texttt{decrease-key} is given by $O(\log{n})$

\end{document}
